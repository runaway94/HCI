\section{Competitive Analysis}\label{Competitive Analysis}

\subsection{Analysed Competitors}

\begin{itemize}
	\item eBay Kleinanzeigen
	\item Airbnb
\end{itemize}

\subsection{Used scenarios}

\begin{itemize}
	\item Make an offer
	\item Search for an offer
	\item Search through own offers
	\item Contact an User
\end{itemize}

\subsection{Competitive Analyses}

	\subsubsection{Analysis of eBay Kleinanzeigen}
		eBay Kleinanzeigen is a plattform, where you can create small advertisements free of charge and search for small advertisments in lot of diverse categories. Every ad is visible for all users and can still be edited after creation.
		
		\paragraph{Scenarios:}
		
		\paragraph{Make an offer} 
		Offers can be made with the button "Anzeige aufgeben". It's located in the green banner at the top of the page, right next to the search bar. On the ad creation page they give you tips to make the ad better and make it more legitimate.
		
		\paragraph{Search for an offer}
		To search for an offer, you use the search bar in the green banner at the top of the page. Here you can also choose a category and where you want to search for the item based on a place and a radius.
		
		\paragraph{Search through own offers}
		To get to your own offers, you have to hover with the mouse over the "Meins" part in the green banner. This will show a drop down menu where you can find an offer entry, which brings you to all your current offers.
		
		\paragraph{Contact a User}
		To contact a user, you can do that at the page of his offer. There is a button to write a message to the user. After clicking the button a new window opens on top of the offer. It shows the username again, has a place where you can write a message and you have the option to also send the user your name and your phone number.
		
		\paragraph{Conclusion:}
		
		\paragraph{Pros}
		\begin{itemize}
			\item well-arranged layout
			\item self-explanatory operability
		\end{itemize}
		
		\paragraph{Cons}
		\begin{itemize}
			\item landing page is way too packed with offers and the categories
			\item almost half of the width of the page is unused
		\end{itemize}
	
	\subsubsection{Analysis of Airbnb}
		Airbnb is a plattform which allows the user to either rent or lease a room, an appartment or a house for a specific time frame and a set amount of money.
		
		\paragraph{Scenarios:}
		
		\paragraph{Make an offer}
		To make an offer you have to become a host first.
		
		\paragraph{Search for an offer}
		At the landing page you are able to make a rather specific search. You need to specify the place, the date when you wanna stay. There is also the possibility to search after cultural events at certain places.
		
		\paragraph{Search through own offers}
		We can say nothing about searching through own offers because nobody in the team is a host on Airbnb.
		
		\paragraph{Contact a User}
		At the end of every page you have a "Contact Host" button, which sends you to a new page. There you have a quick overview over the property and there is a place to write a message.
		
		\paragraph{Pros}
		\begin{itemize}
			\item very well-arranged layout
			\item very good and precise search
		\end{itemize}
	
		\paragraph{Cons}
		\begin{itemize}
			\item The top-most banner at the search changes the color to quick if you start scrolling through the page, creating a dissonance
		\end{itemize}
	
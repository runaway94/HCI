\clearpage
\section{Evaluation}

\subsection{Evaluation from group one}
After the first draft of the prototype was finished, it was evaluated by another team.
What was immediately noted positively was the idea of using different color schemes.
The map used in the search menu to limit the radius was also highlighted. \\
What the team noticed rather negatively was that the advertising between the found items was not marked conspicuously enough. \\
Also some errors in formatting, especially incosistency in icons and the text on the landing page, were chalked up.\
We were able to respond to this criticism immediately and revise the items.\
\\
What was additionally noted was related to the color scheme on the landing page, which is always in the orange renter roll. \\
\\
During the presentation it was also noted that the chat should be offered even more options directly in the chat format. For example, ready-made answers would be a useful addition. The rating was also lost in the detailed view of the products. \\
This criticism would definitely be addressed in the further design phase of the application.


\subsection{Recap}

If you look back at the journey, you can see how useful the individual steps have proven to be.\\
\\
At the beginning, the classification into different user roles provided a very good overview of what should be possible in the application. A user story map offers the possibility to quickly become aware of the potential use cases. \\
Through the constant exchange with the team members, it quickly became clear that there were still some ambiguities. Who can lend and how do you handle private data? What can be lent? Should a sale via the platform also be possible? \\
Through the Journey map, the team was able to clarify these questions early on.\\
\\
Once the journey of both people became clear, it was a matter of figuring out what should be particularly important and where the focus should be. Initial brainstorming and subsequent sorting according to the importance of user stories proved to be very efficient.\\
\\
With the help of research on the web, design guidelines could be found. Together we noted which of these we wanted to implement. This helped both with inspiration and with the creation of initial ideas for the user interface.\\
\\
The usability guidelines also helped in a similar way.\\
\\
For the first time, all participants did the paper prototyping independently of each other. This made it possible for different ideas from several people to flow together. In the following discussion we agreed on different elements. \\
It probably makes sense not to dwell too long on paper prototyping, because otherwise ideas get too complicated and it depends a lot on the first impulse.\\
\\
Designing a prototype in Figma was difficult at first and threatened to be time-consuming. However, with practice and some tutorial videos, the handling became easier and Figma turned out to be a very helpful tool.\\
\\
\\
In the end, it can be said that different opinions in the team led to a detailed discussion of the topic and a good solution was found. //
It was very exciting to see how an idea was gradually turned into a finished prototype through the many steps.




